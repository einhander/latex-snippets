
\begin{center}
\begin{longtable}{m{10cm}cc}
\caption{Характеристика гемосорбентов КАУ}
\label{tabular:kau} \\
\hline
\multirow{2}*{Показатели}&Марка\\
\cline{2-3}
&КАУ-1&КАУ-2\\
\hline
\multicolumn{1}{c}{1}&\multicolumn{1}{c}{2}&\multicolumn{1}{c}{3}\\
\hline \endfirsthead% здесь заканчивается “главный” заголовок таблицы
\hline
\multicolumn{1}{c}{1}&\multicolumn{1}{c}{2}&\multicolumn{1}{c}{3}\\
\hline \endhead

1.Суммарный объем пор, $\textit{см}^3/\textit{г}$, не менее&1,2&1.6 \\
2.Объем сорбционных пор, $\textit{см}^3/\textit{г}$, не менее&0,4&0.6 \\
3.Удельная поверхность, $\textit{м}^2/\textit{г}$, не менее&400&600 \\
4.Механическая прочность на истирание, \%, не ниже&80&65\\
5. Клиренс к 60 мин перфузии от исходного по веществам - маркерам, \% не менее:&&\\
по мединалу&55& \\
креатинину&60& \\
6. Клиренс по веществам средней молекулярной массы в течении первого часа перфузии в мл/мин, не менее&
50& \\
7. Падение уровня тромбоцитов в течении первого часа перфузии по отношению к гепатитовому фону, \% не более&20& \\
\hline
\end{longtable}
\end{center}