
%============Химия=========================================================================
\usepackage{chemfig,mhchem} % рисование структурных формул в химии <<настоящий ад и вынос мозга>>
\makeatletter %использование mhchem для строчных атомов в структурных формулах
\def\CF@node@content{%
	\expandafter\expandafter\expandafter
	\printatom\expandafter\expandafter\expandafter
	{\csname atom@\number\CF@cnt@atomnumber\endcsname}%
	\ensuremath{\CF@node@strut}%
}
\makeatother
%	\setcrambond{2pt}{}{} % для размера треугольных связей
%	\setatomsep{2.2em} % для размера связей между атомами
%	\setbondstyle{semithick}
%\setarrowoffset{0.25em} %расстояние между стрелкой и атомом, полезно для диаграмм
\setdoublesep{0.35700 em}  % 'Bond Spacing'
%\setatomsep{0.78500 em}
\setatomsep{1.78500 em}    % 'Fixed Length'
\setbondoffset{0.18265 em} % 'Margin Width'
\newcommand{\bondwidth}{0.06642 em} % 'Line Width'
\setbondstyle{line width = \bondwidth}
\renewcommand*{\printatom}[1]{{\sffamily\cf{#1}}}
%============================Позволяет рисовать полимеры===========
\newcommand\setpolymerdelim[2]{\def\delimleft{#1}\def\delimright{#2}}
\def\makebraces[#1,#2]#3#4#5{%
	\edef\delimhalfdim{\the\dimexpr(#1+#2)/2}%
	\edef\delimvshift{\the\dimexpr(#1-#2)/2}%
	\chemmove{%
		\node[at=(#4),yshift=(\delimvshift)]
		{$\left\delimleft\vrule height\delimhalfdim depth\delimhalfdim
			width0pt\right.$};%
		\node[at=(#5),yshift=(\delimvshift)]
		{$\left.\vrule height\delimhalfdim depth\delimhalfdim
			width0pt\right\delimright_{\rlap{$\scriptstyle#3$}}$};}}
%=============Пример==================================
%%\setpolymerdelim()% выбор типа скобок
%%Polyéthylène:
%%\chemfig{\vphantom{CH_2}%используется для поднятия связи на нормальное расстояние от базовой линии
%%	-[@{op,.75}]CH_2-CH_2-[@{cl,0.25}]}
%%\makebraces[5pt,5pt]{\!\!n=13}{op}{cl}
%===============конец химии=======================

%конец химии
