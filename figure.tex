%==============один рисунок=====================
\begin{figure}[h]
	\center{\includegraphics[width=0.7\linewidth]{skt}}
	\caption{Схема цепи аппаратов производства активных углей типа СКТ \\
1-расходный бункер}
	\label{ris:skt}
\end{figure}
%options page=xx, angle=xx, keepaspectratio
%trim option's parameter order: left bottom right top
%\includegraphics[width=\columnwidth,page=11,clip=true,trim= 10cm 3.5cm 5cm 10cm]{Pinibriket-CFN.pdf}
%=================================================

%=========Включение PDF несколько страниц==========
\usepackage[final]{pdfpages}

	\includepdf[pages={4,5},landscape,pagecommand={
					\thispagestyle{plain},
					\label{ris:ris1} 
					},templatesize={180mm}{160mm},angle=270]{bio-diesel}
%==================================================		
		


%два рисунка с одной подписью и буквами
\begin{figure}[h]
	\begin{minipage}[h]{0.49\linewidth}
		\center{\includegraphics[width=0.5\linewidth]{image} \\ а)}
	\end{minipage}
	\hfill
	\begin{minipage}[h]{0.49\linewidth}
		\center{\includegraphics[width=0.5\linewidth]{image} \\ б)}
	\end{minipage}
	\caption{Зависимость сигнала от шума для данных.}
	\label{ris:image1}
\end{figure} 

%А теперь Чччеттыыырре!

\begin{figure}[H]
	\begin{minipage}[h]{0.47\linewidth}
		\center{\includegraphics[width=1\linewidth]{image}} a) \\
	\end{minipage}
	\hfill
	\begin{minipage}[h]{0.47\linewidth}
		\center{\includegraphics[width=1\linewidth]{image}} \\b)
	\end{minipage}
	\vfill
	\begin{minipage}[h]{0.47\linewidth}
		\center{\includegraphics[width=1\linewidth]{image}} c) \\
	\end{minipage}
	\hfill
	\begin{minipage}[h]{0.47\linewidth}
		\center{\includegraphics[width=1\linewidth]{image}} d) \\
	\end{minipage}
	\caption{Correlation signal peaks: a) numerical experiment, b)
		registered correlation signals, c) intensity distribution of correlation
		signals in numerical experiment, d) correlation signals intensity
		distribution for DCRAW processed data.}
	\label{ris:experimentalcorrelationsignals}
\end{figure}

% два с разными подписями
\begin{figure}[h]
	\begin{center}
		\begin{minipage}[t]{0.4\linewidth}
			\includegraphics[width=1\linewidth]{image}
			\caption{Исходное изображение.} %% подпись к рисунку
			\label{ris:experimoriginal} %% метка рисунка для ссылки на него
		\end{minipage}
		\hfill 
		\begin{minipage}[t]{0.4\linewidth}
			\includegraphics[width=1\linewidth]{image}
			\caption{Закодированное изображение.}
			\label{ris:experimcoded}
		\end{minipage}
	\end{center}
\end{figure}

%три рисука с буквами и на каждый можно ссылаться и одной подписью
%в преамболу
\usepackage{subfigure}

%и далее вставляем этот код: 

\begin{figure}[ht!]  
	\vspace{-4ex} \centering \subfigure[]{
		\includegraphics[width=0.25\linewidth]{actuatorscouplingSheme_decoupledcase.eps} \label{fig:actuatorscouplingSheme_decoupledcase} }  
	\hspace{4ex}
	\subfigure[]{
		\includegraphics[width=0.25\linewidth]{actuatorscouplingSheme_nearestcoupledcase.eps} \label{fig:actuatorscouplingSheme_nearestcoupledcase} }
	\hspace{4ex}
	\subfigure[]{ \includegraphics[width=0.24\linewidth]{actuatorscouplingSheme_nearestcoupled_and_diag_case.eps} \label{fig:actuatorscouplingSheme_nearestcoupled_and_diag_case} }  
	\caption{Coupling cases for the DM models: \subref{fig:actuatorscouplingSheme_decoupledcase} decoupled case; \subref{fig:actuatorscouplingSheme_nearestcoupledcase} coupling between the closest neighbours; \subref{fig:actuatorscouplingSheme_nearestcoupled_and_diag_case} coupling between the closest neighbour and diagonally adjacent actuators.} \label{fig:threeDMcases}
\end{figure}

%Рисунок с обтеканием
\usepackage{wrapfig}

\begin{wrapfigure}[16]{R}{0.5\linewidth} 
	\vspace{-5ex}
	\includegraphics[width=\linewidth]{image}
	\caption{Some caption}
	\label{fig:somelabel}
\end{wrapfigure}
%Подсвеченные параметры означают:  
%[16] - определяет высоту рисунка в число строк текста и позволяет отбить дополнительное место для рисунков.  
%{r} - положение картинки на странице, можно слева {l} или справа {r}, R или L не позволят рисунку зайти за границу страницы
%I и O - ближе к переплету или дальше для двух-страничного документа
%{0.5\linewidth} - ширина картинки в относительных единицах от ширины линии.
%Тонкий момент: на самом деле, параметр положения рисунка на странице можно писать с маленькой буквы (слева {l} или справа {r}), но это заставляет ЛаТеХ поместить изображение именно там, где вы прикажете. Если же ставить большие буквы (слева {L} или справа {R}), то это даст больше свободы ЛаТеХу.
%Ещё тонкость в том, что автоматика может сделать слишком много (или мало) места в верхней или нижней части обтекаемого рисунка. Вот тут нам поможет дополнительный аргумент [lineheight], который в данном примере задан как [16]. Он-то определяет высоту рисунка в строках текста. Другой возможностью является добавление или удаления промежутков с помощью команды \vspace и данном примере \vspace{-5ex} ЛаТеХу приказывается сместить картинку немного вверх, чтобы выиграть место для текста. 

% Фильмы
\movie[width=\textwidth,height=0.5625\textheight]{\hspace{0.3\textwidth}\includegraphics[width=0.25\columnwidth]{play}}{images/extruder.mp4}

% Рисунок на отдельной странице при заполнении предыдущей
\afterpage{%
	\clearpage% Flush earlier floats (otherwise order might not be correct)
	\begin{landscape}
		\begin{figure}[b]
			\center{\includegraphics[width=\linewidth]{skt}}
			\caption[Схема цепи аппаратов производства активных углей типа СКТ.]{Схема цепи аппаратов производства активных углей типа СКТ.\\
				1 --- расходный бункер; 2 --- валковая дробилка; 3 --- молотковая дробилка; 4 --- тарельчатый питатель; 5 --- вибрационный грохот; 6;9 --- бункерные весы; 7 --- смеситель; 8 --- бак приготовления раствора K\textsubscript{2}S; 10 --- вертикальный пресс; 11;18 --- сушильный барабан; 12 --- секторный питатель; 13 --- печь карбонизации; 14 --- печь активации; 15 --- барабан охлаждения; 16 --- узел выщелачивания; 17 --- узел кислотной отмывки; 19 --- прокалочная печь; 20 --- сборник АУ.}
			\label{ris:skt}
		\end{figure} 
	\end{landscape}
}

%======================  вставка pdf  ==================
\includepdf[pages=1,
    noautoscale=false,
    scale=1.093,
    pagecommand={
        \thispagestyle{plain}
        \section*{Приложение 1 Протоколы исследований}
        \label{ris:ris1}  
},templatesize={210mm}{297mm}]{moisture.pdf}
\includepdf[pages=2-,
    noautoscale=false,
    scale=1.093,
    pagecommand={
        \thispagestyle{plain}
        %\section{Приложение}
        \label{ris:ris1}  
},templatesize={210mm}{297mm}]{moisture.pdf}

\includepdf[pages=-,
    noautoscale=false,
    scale=1.093,
    pagecommand={
        \thispagestyle{plain}
        %\section{Приложение}
        \label{ris:ris1}  
},templatesize={210mm}{297mm}]{ash.pdf}

