%создаем специальный блок другого цвета
\newenvironment<>{problock}[1]{%
	\begin{actionenv}#2%
		\def\insertblocktitle{#1}%
		\par%
		\mode<presentation>{%
			\setbeamercolor{block title}{fg=white,bg=orange!20!black}
			\setbeamercolor{block body}{fg=black,bg=olive!50}
			\setbeamercolor{itemize item}{fg=orange!20!black}
			\setbeamertemplate{itemize item}[triangle]
		}%
		\usebeamertemplate{block begin}}
	{\par\usebeamertemplate{block end}\end{actionenv}}


%цвет блока для всего документа
%\setbeamercolor{block title}{fg=orange!20!black,bg=green}%bg=background, fg= foreground
%\setbeamercolor{block body}{bg=yellow,fg=green}%bg=background, fg= foreground


% If you have a file called "university-logo-filename.xxx", where xxx
% is a graphic format that can be processed by latex or pdflatex,
% resp., then you can add a logo as follows:

\logo{\includegraphics[height=1cm]{lta}}

\usetheme{Frankfurt} %тема
%Хорошие темы: Frankfurt - в верху есть очень красивый прогерссбар 
%\usetheme{Warsaw} %тема
%\usecolortheme{wolverine} %цвет

%\usecolortheme[overlystylish]{albatross}  %цвет внешних элементов темы
%\usecolortheme{lily} %цвет внутренних элементов темы


\setbeamercovered{transparent} %полупрозрачные оверлеиы




%параметры ссылок и документа
\hypersetup{colorlinks=true,
	pdfpagemode={FullScreen},
	linkcolor=blue,
	citecolor=blue,
	filecolor=blue,
	urlcolor=blue,
	pdfauthor=Андрей Спицын,
	pdfsubject=Пиролиз древесины, 
	pdfkeywords={Пиролиз,Pyrolisys,adsorbtion}
}

%относится к первому слайду
\title{Курсовой проект}
\subtitle{Новые направления в химии и биотехнологии лесохимических продуктов}
\author{Спицын Андрей Александрович}
\institute{Санкт-Петербургский государственный лесотехнический университет}
\date{\today}
