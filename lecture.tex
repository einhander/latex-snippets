\documentclass[pdf,hyperref={unicode}]{beamer}
\usepackage[T2A]{fontenc}
\usepackage[utf8]{inputenc}
\usepackage[english,russian]{babel}
\usepackage{amssymb,amsfonts,amsmath,mathtext,textcomp}
\usepackage{cite,enumerate,float,indentfirst,pgf}


%\usepackage{pdfcomment} %коментарии в pdf
%\usepackage{nemonotes} % заметки в pdf
%\usepackage[dvips]{graphicx}
\usepackage{hyperref} %красивые ссылки и параметры pdf
\hypersetup{colorlinks=true,
		pdfpagemode={FullScreen},
		linkcolor=blue,
		citecolor=blue,
		filecolor=blue,
		urlcolor=blue,
		pdfauthor=Андрей Спицын,
		pdfsubject=Пиролиз древесины, 
		pdfkeywords={Пиролиз,Pyrolisys,adsorbtion}
		} %параметры ссылок и документа
\usepackage{graphicx} %хотим вставлять в рисунки?
%\usepackage[dvips]{graphicx}
\graphicspath{{images/}}%путь к рисункам
\setbeamercovered{highly dynamic} %оверлеиы transparent= dynamic highly dynamic



%создаем специальный блок другого цвета
\newenvironment<>{problock}[1]{%
  \begin{actionenv}#2%
      \def\insertblocktitle{#1}%
      \par%
      \mode<presentation>{%
        \setbeamercolor{block title}{fg=white,bg=orange!20!black}
       \setbeamercolor{block body}{fg=black,bg=olive!50}
       \setbeamercolor{itemize item}{fg=orange!20!black}
       \setbeamertemplate{itemize item}[triangle]
     }%
      \usebeamertemplate{block begin}}
    {\par\usebeamertemplate{block end}\end{actionenv}}


%цвет блока для всего документа
%\setbeamercolor{block title}{fg=orange!20!black,bg=green}%bg=background, fg= foreground
%\setbeamercolor{block body}{bg=yellow,fg=green}%bg=background, fg= foreground


% If you have a file called "university-logo-filename.xxx", where xxx
% is a graphic format that can be processed by latex or pdflatex,
% resp., then you can add a logo as follows:
 
\logo{\includegraphics[height=1cm]{lta}}

\usetheme{Frankfurt} %тема
%Хорошие темы: Frankfurt - в верху есть очень красивый прогерссбар 

%\usecolortheme[overlystylish]{albatross}  %цвет внешних элементов темы
%\usecolortheme{lily} %цвет внутренних элементов темы


%относится к первому слайду
\title[Лекция 8 и 9]{ВЫДЕЛЕНИЕ, СБОР И ПЕРЕРАБОТКА СУЛЬФАТНОГО МЫЛА}


\subtitle{Рациональное использование природных ресурсов \\
Лекции 8 и 9}
\author[Спицын~А.~А.]{Спицын Андрей Александрович}
\institute[СПбГЛТУ]
{Санкт-Петербургский государственный лесотехнический университет им. С.~М.~Кирова}
\date{\today}

\begin{document}
%%титульная страница
\maketitle
%% основной текст
\section{Сульфатная варка древесины}
\subsection{Сульфатная варка древесины}
\begin{frame}[c]{Схема сульфатной варки древесины}
\transdissolve[duration=0.2]
\includegraphics[width=0.8\columnwidth]{sulphate-pulping}
\end{frame}

\begin{frame}[c]{Процесс подготовки сырья}
\transdissolve[duration=0.2]
\begin{center}
\begin{columns}
       \column{0.49\textwidth}
       \begin{itemize}
       \item распиловка
       \item окорка
       \item рубка в щепу и последующая её сортировка
       \end{itemize}
        \column{0.49\textwidth}
        \includegraphics[width=0.8\columnwidth]{800px-Holzhackschnitzel}
\end{columns}
\end{center}
\end{frame}


\begin{frame}[c]{Регенерация щелока}
\transdissolve[duration=0.2]
\begin{center}
\includegraphics[width=0.5\columnwidth]{09}
\begin{columns}
       \column{0.4\textwidth}
       \begin{block}{Получение зеленого щелока}
              \begin{equation}
              Na_2SO_4 + 4C = Na_2S + 4CO
              \end{equation}
              \begin{equation}
              2NaOH + CO_2 = Na_2CO_3 + H_2O
              \end{equation}
       \end{block}
       \column{0.59\textwidth}
       \begin{block}{Получение извести}
                \begin{equation}
                Na_2CO_3 + Ca(OH)_2 = 2NaOH + CaCO_3\downarrow
                \end{equation}
                \begin{equation}
                CaCO_3 = CaO + CO_2
                \end{equation}
                \begin{equation}
                CaO + H_2O = Ca(OH)_2
                \end{equation}
        \end{block}
\end{columns}
\end{center}
\end{frame}

\begin{frame}[c]{Белый щелок}
\transdissolve[duration=0.2]
\includegraphics[width=0.8\columnwidth]{valkolip_koostumus}
\end{frame}

\begin{frame}[c]{Черный щелок}
\transdissolve[duration=0.2]
\includegraphics[width=0.8\columnwidth]{mustalip_koost}
\end{frame}

\begin{frame}[c]{Зеленый щелок}
\transdissolve[duration=0.2]
\includegraphics[width=0.8\columnwidth]{viherlipea_analyysi}
\end{frame}

\begin{frame}[c]{Котел переодической варки}
\transdissolve[duration=0.2]
\includegraphics[width=0.8\columnwidth]{erakeitin}
\end{frame}

\begin{frame}[c]{Завод переодической варки}
\transdissolve[duration=0.2]
\includegraphics[width=0.8\columnwidth]{SB_keittamo}
\end{frame}


\begin{frame}[c]{Процесс непрерывной варки}
\transdissolve[duration=0.2]
\begin{center}
\begin{columns}
       \column{0.74\textwidth}
       \only<1>{
       \begin{block}{Основными достоинствами процесса являются:}
       \begin{itemize}
       \item исключение из процесса технологических операций по загрузке и выгрузке варочных котлов;
       	\item    сокращение производственных и складских площадей;
        \item     меньший расход теплоносителей и их стабильное потребление во времени;
        \item              сокращение расходов на теплорекуперацию;
        \item              возможность полной автоматизации процессов
       \end{itemize}
       \end{block}
       }
      \pause
      \only<2>{
      \begin{alertblock}{Основными недостатками и особенностями процесса являются:}
      \begin{itemize}
       \item              метод может быть эффективно применим только для крупнотоннажного производства;
        \item      более высокие требования к качеству исходного сырья, стабильности параметров пара и варочного раствора;
        \item      более технологически сложная эксплуатация оборудования.
      \end{itemize}
      \end{alertblock}
      }
      
        \column{0.25\textwidth}
        \flushleft\includegraphics[width=0.99\columnwidth]{kamyr_valokuva}
\end{columns}
\end{center}
\end{frame}


\section{Выделение сырого сульфатного мыла}
\subsection{Сырое сульфатное масло}
\begin{frame}[c]{Сырое сульфатное мыло}
\transdissolve[duration=0.2]
\begin{block}{}
Сырое сульфатное мыло представляет собой смесь, состоящую из приблизительно равных количеств натриевых солей смоляных и жирных кислот, и сравнительно меньшего количества окисленных и неомыляемых веществ. 
\end{block}
%\pause
%\begin{alertblock}
%content...
%\end{alertblock}
%\pause
%\begin{problock}
%content...
%\end{problock}

\end{frame}



\begin{frame}[c]{Загрязнители сульфатного мыла}
\transdissolve[duration=0.2] %переход к другому элементу ставится после названия слайда
%These transitions are available.
%\transblindshorizontal				Horizontal blinds pulled away
%\transblindsvertical				Vertical blinds pulled away
%\transboxin						Move to center from all sides			[duration=0.2]	
%\transboxout						Move to all sides from center
%\transdissolve						Slowly dissolve what was shown before
%\transglitter			(off)		Glitter sweeps in specified direction 	[direction=<degree>]
%\transslipverticalin	(off)		Sweeps two vertical lines in
%\transslipverticalout				Sweeps two vertical lines out
%\transhorizontalin					Sweeps two horizontal lines in
%\transhorizontalout				Sweeps two horizontal lines out
%\transwipe							Sweeps single line in specified direction
%\transduration{2}					Show slide specified number of seconds
\begin{block}{}
В составе загрязнений находится значительное количество лигнина, минеральных и других веществ, содержащихся в черном щелоке, увлекаемом при отстаивании выделяющимся сульфатным мылом.

\end{block}

\end{frame}





\begin{frame}[c]{Состав сырого сульфатного мыла}
\transdissolve[duration=0.2]
\begin{table}[h] %h - таблица где-то здесь, H - таблица здесь и баста
\begin{center}
\begin{tabular}{lc}
\hline
смоляные и жирные	 кислоты, \%&45 — 55\\
неомыляемые вещества, \%&4 — 8 \\
лигнин, красящие и другие вещества, \% &1 — 3\\
натрий в соединениях, \%&4 — 8\\
свободная щелочь, сульфат и карбонат, \% &1—3\\ 
вода, \% &30—35\\
\hline
\end{tabular}
\end{center}
\end{table}
\end{frame}

\begin{frame}[c]{Теоретический выход мыла }
\transdissolve[duration=0.2]
\begin{itemize}			%обязательно окружение для создания колонок
\item при смолистости сырья около 3~\%  приблизительно 120 — 130 кг/т целлюлозы
\item при смолистости 4,5~\%— до 180 — 200 кг/т целлюлозы.
\end{itemize}
\end{frame}


\begin{frame}[c]{Технологические параметры, определяющие процесс выделения сульфатного мыла}
\transdissolve[duration=0.2]
\begin{itemize}
\item плотность черного щелока
\item  продолжительность его отстаивания
\item  температура процесса
\end{itemize}
\end{frame}

\begin{frame}[c]{Схема выделенния сульфатного мыла}
\transdissolve[duration=0.2]
\begin{center}
\includegraphics[width=0.8\columnwidth]{shema}
\end{center}
\end{frame}

\begin{frame}[c]{Способы интенсификации выделения сульфатного мыла}
\begin{itemize}
\item Добавка электролитов к черным щелокам \pause
\item Методы флотации и аэрации \pause
\item Использование добавок органических веществ\pause
\item Использование солей первичных алифатических аминов \pause
\item Применение электрофлокуляции \pause
\item Окисление черных щелоков \pause
\item Другие
\end{itemize}
\end{frame}

\section{Литература}
\begin{frame}[c]{Литература}
\transdissolve[duration=0.2]
\begin{center}
Переработка сульфатного и сульфитного Щелоков: Учебник для вузов / Б. Д. Богомолов, С. А. Сапотницкий, О. М. Соколов и др.— М.: Лесная промышленность, 1989 - 360 с.
\end{center}
\end{frame}


\end{document}
