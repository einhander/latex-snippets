\documentclass[russian,utf8]{eskdtext}
%\usepackage{ifxetex}
%\ifxetex
	\usepackage{fontspec}
	\usepackage{xunicode}
	\usepackage{xltxtra}
	\defaultfontfeatures{Mapping=tex-text, Scale=MatchLowercase}
	\newfontfamily{\cyrillicfont}{FreeSerif}
	\setmainfont{FreeSerif}
	\setromanfont{FreeSerif}
	\usepackage{polyglossia}
	\setdefaultlanguage{russian}
%\else
%	\usepackage[T2A]{fontenc} 
%	\usepackage[utf8]{inputenc}
%	\usepackage[russian]{babel}
%\fi
\ESKDdepartment{Общество с Ограниченной Ответственностью}
\ESKDcompany{Пианга РУС}
\ESKDclassCode{025149}
\ESKDtitle{Анализ ГПМ,УМП,АЖКТ,ПГС.}
\ESKDdocName{Техническое задание на анализ}
\ESKDsignature{ОПМУПП 00.00.00.00.00.00 ТЗ}
\ESKDauthor{Белоусов~И.~И.}
\ESKDtitleApprovedBy{ Директор ООО "Пианга РУС"}{Выскребенцев~И.~Ю.}
\ESKDtitleDesignedBy{ Главный инженер ООО "Пианга РУС"}{Амфилохиев~Б.~Л.}
\ESKDtitleAgreedBy{ Заместитель директора ФГУП "ВНИИМ им.Д.И.Менделеева"}{Кривцов~Е.~П.}
\ESKDtitleAgreedBy{ Руководитель лаборатории калориметрии ФГУП "ВНИИМ им.Д.И.Менделеева"}{Корчагина~Е.~Н.}
\ESKDtitleDesignedBy{ Ведущий инженер ООО "Пианга РУС"}{Белоусов~И.~И.}
\ESKDtitleDesignedBy{ Главный технолог разработки, инженер ООО "Пианга РУС"}{Литвинов~В.~В.}

\begin{document}
\maketitle
\tableofcontents
\newpage
\section{Название проводимой работы}
Анализ состава и физико-химических свойств гранул продукта жизнедеятельности птиц (ГПМ),углеродно минерального остатка (УМП),альтернативного жидкого котельного топлива (АЖКТ), парогазовой смеси (ПГС).
\section{Основание для проведения работы}
Договор между ООО "Пиянга РУС" и ФГУП "ВНИИМ им.Д.И. Менделеева." далее Договор.
\section{Срок исполнения}
Сроки исполнения не должен превышать двух месяцев с момента предоставления последней пробы.
\section{Область применения результатов научных исследований}
%Актуальность разработки - зачем сделано.
Инициативная разработка Заказчика.

\section{Цель работы}
Анализ состава, физико-химических свойств, теплоемкости, теплотворной способности следующих образцов: гранул продукта жизнедеятельности птиц, углеродно -- минерального продукта, альтернативного жидкого котельного топлива и компонентов летучих продуктов в виде парогазовой смеси. Дополнительно анализ второй фракции жидкого продукта (далее Вторая фракция).

\section{Краткое описание работы}
Образцы ГПМ влажностью 8-12\% диаметром 6-8 мм длиной до 30 мм.Образцы	УПМ - углеродно-минеральный продукт в виде мелко кускового угля. Аналог: древесный
уголь с высоким содержанием минеральных компонентов.
   
Образцы АЖКТ - альтернативное жидкое котельное топливо. Аналог: пирогенная древесная смола.
   
Образцы ПГС - частично очищенная паро-газовая смесь с содержанием взвеси жидких продуктов. Аналог: низкокалорийный газ коксо-доменных печей.

\section{Основные параметры и технические требования}
\subsection{Общие требования к работе}
Выполнение работ должно осуществляться в соответствии с требованиями действующего законодательства, федеральных, региональных, отраслевых и (или) ведомственных норм и правил, государственных стандартов и технических регламентов, правил устройства и эксплуатации.
\subsection{Задачи, решаемые в ходе работы}
Определение следующих показателей для образцов:
\begin{enumerate}
\item[1] ГПМ:
   \begin{enumerate}
   \item[1.1] Массовая доля воды, \%
   \item[1.2] Зольность, \%
   \item[1.3] Теплоемкость от 20 до 600 С\textsuperscript{o}, с шагом 10\textsuperscript{o} при скорости нагрева 1\textsuperscript{o} в минуту, кДж/кг*K
  % \item[1.4] Температура вспышки, С\textsuperscript{o}. не могут, только жидкие где вспыхивают пары
   \item[1.4] Теплота сгорания высшая/низшая в пересчете на сухое топливо, кДж/кг
 %  \item[1.6] Временное сопротивление разрушению гранул разной влажности от половины исходной до а.с.д. кгс/см\textsuperscript{2}. нет оборудования - нужны дорожники и прочие.
   \item[1.5] Элементный состав, в том числе:
   \begin{itemize}
         \item углерод органический
         \item углерод неорганический
        % \item общий азот                    
         \item общий фосфор (в пересчете на $P_{2}O_{5}$)
         \item общий калий (в пересчете на  $K_{2}O$)
        % Дополнительно согласно требованиям СанПиН. 2.1.7.573-96.
            \item мышьяк
            \item кадмий
            \item хром 
            \item медь
            \item ртуть
            \item марганец 
            \item никель 
            \item свинец 
            \item цинк
            \item кальций 
            \item натрий 
            \item магний 
            \item железо
            \end{itemize}
      \end{enumerate}
      
   \item[2] АЖКТ:
      \begin{enumerate}
      \item[2.1] Вязкость кинематическая при 90 С\textsuperscript{o}, мм\textsuperscript{2}/с 
      \item[2.2] Плотность, кг/см\textsuperscript{2}
      \item[2.3] Температура застывания, С\textsuperscript{o}
      \item[2.4] Теплота сгорания высшая/низшая в пересчете на сухое топливо, кДж/кг
      \item[2.5] Зольность, \%
      \item[2.6] Температура вспышки, С\textsuperscript{o}.
      \item[2.7] Теплоемкость от 90 до температуры вспышки образца с шагом 25\textsuperscript{o}, кДж/кг*K
      \item[2.8] Коксуемость \%
      \item[2.9] Водородный показатель агрессивности среды (Показатель кислотности), pH
      \item[2.10] Фракционный состав (начальная и конечная температура кипения, процент отгона)
      \item[2.11] Элементный состав, в том числе:
      \begin{itemize}
            \item углерод органический
            \item углерод неорганический
        %    \item общий азот
            \item общий фосфор (в пересчете на  $P_{2}O_{5}$)
            \item общий калий (в пересчете на  $K_{2}O$)
            %Дополнительно по требованиям СанПиН. 2.1.7.573-96
            \item мышьяк  
            \item кадмий
            \item хром 
            \item медь 
            \item ртуть
            \item марганец 
            \item никель 
            \item свинец 
            \item цинк 
      		\end{itemize}
      \end{enumerate}
      
   \item[3] УМП:
      \begin{enumerate}
      \item[3.1] Зольность, \%
      \item[3.2] Содержание летучего/нелетучего углерода, \%
    %  \item[3.3] Температура вспышки (С\textsuperscript{o}) не могут, только жидкие где вспыхивают пары
      \item[3.3] Теплоемкость от 20 до 600 С\textsuperscript{o} с шагом 25\textsuperscript{o}, кДж/кг*K
      \item[3.4] Теплота сгорания высшая/низшая в пересчете на сухое топливо, кДж/кг
   %   \item[3.6] Временное сопротивление разрушению гранул разной влажности от половины исходной до а.с.д. кгс/см\textsuperscript{2} нет оборудования
      \item[3.5] Элементный состав, в том числе:
         \begin{itemize}
    %     \item общий азот
         \item общий фосфор (в пересчете на  $P_{2}O_{5}$)
         \item общий калий (в пересчете на  $K_{2}O$)
         %Дополнительно по требованиям СанПиН. 2.1.7.573-96
         \item мышьяк
         \item кадмий
         \item хром
         \item медь
         \item ртуть
         \item марганец
         \item никель
         \item свинец
         \item цинк
         %Дополнительно по предыдущему исследованию уточнить:
         \item кальций"
         \item натрий
         \item магний
         \item железо
         \end{itemize}
      \end{enumerate}   
      
   \item[4] ПГС:
      \begin{enumerate}
      \item[4.1] Теплота сгорания высшая/низшая в пересчете на сухое топливо, кДж/м\textsuperscript{3}
    %  \item[4.2] Температура вспышки, C\textsuperscript{o} не могут не знают как.
    %  \item[4.3] Теплоемкость от 100 до 600 С\textsuperscript{o} с шагом 25\textsuperscript{o}, кДж/м\textsuperscript{3}*K\textsuperscript{o} не могут не знают как.
      \item[4.3] Элементный состав, в том числе:
      	\begin{itemize}
      	\item вода
      	\item органические вещества
      	\item кислоты
      	\item спирты 
      	\item $H_{2}$
      	\item $CO_{2}$
      	\item CO
      	\item $CH_{4}$
      	\item $NO_{X}$
      	%Дополнительно по требованиям СанПиН. 2.1.7.573-96
      	\item мышьяк
      	\item кадмий
      	\item хром
      	\item медь
      	\item ртуть
      	\item марганец
      	\item никель
      	\item свинец
      	\item цинк
         \end{itemize}
      \end{enumerate} 
      
   \item[5] Вторая фракция:
      \begin{enumerate}
      \item[5.1] Водородный показатель агрессивности среды (Показатель кислотности), pH
      \item[5.2] Фракционный состав (начальная и конечная температура кипения, процент отгона)
      \item[5.3] Элементный состав, в том числе:
         \begin{itemize}
         \item углерод органический
         \item углерод неорганический
       %  \item общий азот
         \item общий фосфор (в пересчете на  $P_{2}O_{5}$)
         \item общий калий (в пересчете на  $K_{2}O$)
         %Дополнительно по требованиям СанПиН. 2.1.7.573-96
         \item мышьяк  
         \item кадмий
         \item хром 
         \item медь 
         \item ртуть
         \item марганец 
         \item никель 
         \item свинец 
         \item цинк 
         \end{itemize}
      \end{enumerate}
   \end{enumerate}
   \subsection{Требования к предоставлению результатов работы}
   Результат работы предоставляется Заказчику в виде отчета с результатами экспертиз.
      
   В отчете по работе должны быть предоставлены ссылки на методики и стандарты испытаний по показателям соответственно (ATSM, ГОСТ, DIN). 
   
   
   \section{Потребность в результатах работы (планируемые направления применения разработки)}
   Инициативная разработка Заказчика.
   %ГОРТ Р 15.201 - 2000 инициативные разработки продукции без конкретного заказчика при коммерческом риске разработчика и" изготовителя.
   \section{Стадии и этапы разработки}
   %Таблица номер, наименование этапов работ, сроки выполения, научные результаты, научно-техническая продукция и документация, подлежащая сдаче. по стадиям разработки.
   Согласовываются Исполнителем и Заказчиком, регламентируются Договором.
   \section{Требования по защите результатов работы}
   % как исполнитель бужет защищять результаты нир перед заказчиком.
   В отчете должны быть предоставлены ссылки на методики и стандарты испытаний по показателям соответственно (ATSM, ГОСТ, DIN). 
   \section{Результаты работы}
   Результат работы должен быть предоставлен Заказчику в виде отчета с результатами анализов.
   
   В состав отчета должны войти следующие данные о показателях для образцов: 
   \begin{enumerate}
   \item[1] ГПМ:
      \begin{enumerate}
      \item[1.1] Массовая доля воды, \%
      \item[1.2] Зольность, \%
      \item[1.3] Теплоемкость от 20 до 600 С\textsuperscript{o}, с шагом 10\textsuperscript{o} при скорости нагрева 1\textsuperscript{o} в минуту,  С\textsuperscript{o}
     % \item[1.4] Температура вспышки, кДж/кг*K\textsuperscript{o}
      \item[1.4] Теплота сгорания высшая/низшая в пересчете на сухое топливо, кДж/кг
    %  \item[1.6] Временное сопротивление разрушению гранул разной влажности от половины исходной до а.с.д. кгс/см\textsuperscript{2}.
      \item[1.5] Элементный состав, в том числе:
      \begin{itemize}
            \item углерод органический
            \item углерод неорганический
          %  \item общий азот
            \item общий фосфор (в пересчете на  $P_{2}O_{5}$)
            \item общий калий (в пересчете на  $K_{2}O$)
           % Дополнительно согласно требованиям СанПиН. 2.1.7.573-96.
            \item мышьяк
            \item кадмий
            \item хром 
            \item медь
            \item ртуть
            \item марганец 
            \item никель 
            \item свинец 
            \item цинк
            \item кальций 
            \item натрий 
            \item магний 
            \item железо
            \end{itemize}
      \end{enumerate}
      
   \item[2] АЖКТ:
      \begin{enumerate}
      \item[2.1] Вязкость кинематическая при 90 С\textsuperscript{o}, мм\textsuperscript{2}/с 
      \item[2.2] Плотность, кг/см\textsuperscript{2}
      \item[2.3] Температура застывания, С\textsuperscript{o}
      \item[2.4] Теплота сгорания высшая/низшая в пересчете на сухое топливо, кДж/кг
      \item[2.5] Зольность, \%
      \item[2.6] Температура вспышки, С\textsuperscript{o}
      \item[2.7] Теплоемкость от 90 до температуры вспышки образца с шагом 25\textsuperscript{o}, кДж/кг*K 
      \item[2.8] Коксуемость \%
      \item[2.9] Водородный показатель агрессивности среды (Показатель кислотности), pH
     % \item[2.10] Кислотное число, мг KOH/г
      \item[2.10] Фракционный состав (начальная и конечная температура кипения, процент отгона)
      \item[2.11] Элементный состав, в том числе:
      \begin{itemize}
            \item углерод органический
            \item углерод неорганический
          %  \item общий азот
            \item общий фосфор (в пересчете на  $P_{2}O_{5}$)
            \item общий калий (в пересчете на  $K_{2}O$)
            %Дополнительно по требованиям СанПиН. 2.1.7.573-96
            \item мышьяк  
            \item кадмий
            \item хром 
            \item медь 
            \item ртуть
            \item марганец 
            \item никель 
            \item свинец 
            \item цинк 
      		\end{itemize}
      \end{enumerate}
      
   \item[3] УМП:
      \begin{enumerate}
      \item[3.1] Зольность, \%
      \item[3.2] Содержание летучего/нелетучего углерода в \%
      \item[3.3] Температура вспышки, С\textsuperscript{o}
      \item[3.4] Теплоемкость от 20 до 600 С\textsuperscript{o} с шагом 25\textsuperscript{o}, кДж/кг*K
      \item[3.5] Теплота сгорания высшая/низшая в пересчете на сухое топливо, кДж/кг
    %  \item[3.6] Временное сопротивление разрушению гранул разной влажности от половины исходной до а.с.д. кгс/см\textsuperscript{2}
      \item[3.6] Элементный состав, в том числе:
         \begin{itemize}
      %   \item общий азот
         \item общий фосфор (в пересчете на  $P_{2}O_{5}$)
         \item общий калий (в пересчете на  $K_{2}O$)
         %Дополнительно по требованиям СанПиН. 2.1.7.573-96
         \item мышьяк
         \item кадмий
         \item хром
         \item медь
         \item ртуть
         \item марганец
         \item никель
         \item свинец
         \item цинк
         %Дополнительно по предыдущему исследованию уточнить:
         \item кальций
         \item натрий
         \item магний
         \item железо
         \end{itemize}
      \end{enumerate}   
      
   \item[4] ПГС:
      \begin{enumerate}
      \item[4.1] Теплота сгорания высшая/низшая в пересчете на сухое топливо, кДж/м\textsuperscript{3}.
    %  \item[4.2] Температура вспышки, С\textsuperscript{o}. не могут не знают как.
    %  \item[4.3] Теплоемкость от 100 до 600 С\textsuperscript{o} с шагом 25\textsuperscript{o},  кДж/м\textsuperscript{3}*K\textsuperscript{o}  не могут не знают как.
      \item[4.2] Элементный состав, в том числе:
      \begin{itemize}
      \item вода
      \item пары органических веществ
      \item кислоты
      \item спирты 
      \item $H_{2}$
      \item $CO_{2}$
      \item CO
      \item $CH_{4}$
      \item $NO_{X}$
      %Дополнительно по требованиям СанПиН. 2.1.7.573-96
      \item мышьяк
      \item кадмий
      \item хром
      \item медь
      \item ртуть
      \item марганец
      \item никель
      \item свинец
      \item цинк
         \end{itemize}
      \end{enumerate}
      
  \item[5] Вторая фракция:
      \begin{enumerate}
      \item[5.1] Водородный показатель агрессивности среды (Показатель кислотности), pH
      \item[5.2] Фракционный состав (начальная и конечная температура кипения, процент отгона)
      \item[5.3] Элементный состав, в том числе:
         \begin{itemize}
         \item углерод органический
         \item углерод неорганический
      %   \item общий азот
         \item общий фосфор (в пересчете на  $P_{2}O_{5}$)
         \item общий калий (в пересчете на  $K_{2}O$)
         %Дополнительно по требованиям СанПиН. 2.1.7.573-96
         \item мышьяк  
         \item кадмий
         \item хром 
         \item медь 
         \item ртуть
         \item марганец 
         \item никель 
         \item свинец 
         \item цинк 
         \end{itemize}
      \end{enumerate}
   \end{enumerate} 
%   \section{Приложения к ТЗ}
%   Текст текст текст.
   
   \end{document}