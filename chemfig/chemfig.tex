\small
\chemfig{C(-[:-135])(-[:-225])=O}
\chemsign{+}
\chemfig{CH(-[4])=CH-CH=CH-} %мазафака баг если сперва делать связь то молекула опускается \Mad


\chemrel{->}
\chemfig{
	CH(-[1,,1]CH?[,2])(-[4]) % ? - хук для связывания удаленных атомов [имя,тип связи], вторая строчка для отлова бага
	-[7,,1] %направление 1 - 45 град, 2 - 90 град, 3 - 135 град итд.
	C(-[5])(-[6])
	-[0]
	O
	-[1]
	CH(-)
	-[3,,1,1] % связь[направление,,от какого атома, к какому]
	C?[,2]H
	}

а)

\chemfig{CH(-[4])=CH-CH(-[2])=0} %и здесь тоже отлов бага
\chemsign{+} 
\chemfig{CH(-[4])=CH-}
\vspace{1em}


\chemrel{->}
\chemfig{
	CH(-[1,,1]CH?[,2])(-[4]) % ? - хук для связывания удаленных атомов [имя,тип связи], вторые скобки для нейтрализации бага
	-[7,,1] %направление 1 - 45 град, 2 - 90 град, 3 - 135 град итд.
	CH(-[5,,1])
	-
	CH(-[7])
	-[1,,1]
	CH(-)
	-[3,,1,1] % связь[направление,длинна?,от какого атома, к какому]
	C?[,2]H
}

б)

\chemfig{*6(=-(=[7]O)-(=[1]O)-=-)}
\chemsign{+}
\chemfig{[:90]CH(-[::-180])=CH-} %[:90] начальное вращение молекулы
\vspace{1em}


\chemrel{->}
\chemfig{*6(=-(
		-[:-30]O-[:30]C?H-[:-30,,1]
		)
		-(
		-[:30]O-[:-30]C?H-[:30,,1]
		)-=-)}

в)


\scalebox{.6}{
	\schemestart
	Растительная биомасса
	\arrow([fill=blue]--sucr[fill=red])[-90,0.75]
	\parbox{13em}{\centering Сахароза, Крахмал, Целлюлоза, Гемицеллюлоза, Инулин}
	\arrow(--[fill=green])[-135]
	\parbox{6em}{\centering Пентозы \\(ксилоза)}
	\arrow(--[fill=white])[-90]
	\chemfig{*5(1-2=3-4=5-)}
	\arrow(@sucr--[fill=red])[-45]
	\parbox{10em}{\centering Гексозы \\(глюкоза, фруктоза, манноза)}
	\schemestop
}
